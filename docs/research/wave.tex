%% This file is in Public Domain.
\documentclass[11pt]{article}
\usepackage[left=1.5in, right=1.5in, top=1in, bottom=1in]{geometry}
\usepackage{amsmath}
\begin{document}

\title{Harmonic Waves}
\author{Andrew Makousky}
\date{\today}

\maketitle

A harmonic wave takes this form:
$$\sin{x} + \sin{2x} + \sin{3x} + \sin{4x} + \cdots$$
More specifically, the amplitude of each harmonic can be adjusted:
$$f(x) = a\sin{x} + b\sin{2x} + c\sin{3x} + d\sin{4x} + \cdots$$
Given a wave of this form, how do you find the maximum displacement of
the composite wave?  First you would try to take the derivative.
$$f'(x) = a\cos{x} + 2b\cos{2x} + 3c\cos{3x} + 4d\cos{4x} + \cdots$$
The derivative of the harmonic wave function is easily predictable.
Once the derivative is found, the zeroes of the derivative must be
found.  By definition, the zeroes of the cosine function are any
multiples of $\pi$.  However, it may be the case that some terms of
the derivative are positive and negative in such a way that the result
of the whole function is zero.  Is there a method to determine the
exact zeroes of the derivative?

Possible methods to the solution: Fourier Analysis, Fourier Transform,
Laplace Transform.

\end{document}
